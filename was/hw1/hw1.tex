\documentclass[11pt]{article}

\usepackage{amsmath}
\usepackage{amssymb}
\usepackage{amsthm}
\usepackage{enumerate}
\usepackage[norelsize, linesnumbered, ruled, lined, boxed, commentsnumbered]{algorithm2e}
\usepackage{fullpage}

\pdfpagewidth 8.5in
\pdfpageheight 11in 

\topmargin 0in
\oddsidemargin 0in
\evensidemargin 0in

\newtheorem*{claim}{Claim}
\newcommand{\pd}[1]{\frac{\partial}{\partial #1}}

\begin{document}

\begin{center}
    \textbf{CS365 Written Assignment 1} \\
    Khoa Cao \\
    Due 
\end{center}

\section*{Question 1}
\begin{proof}
	First, since we pick the largest number from the $k$ numbers we sampled, if at least one of the $k$ numbers is larger than the median, then the largest number we picked must be larger than the median. \\
	Let $E$ be the event that all $k$ numbers we examined are smaller than or equal to the median of the $n$ numbers. 
	Then, the complement of $E$, denoted $E^c$, is the event that at least one of the $k$ numbers we examined is larger than the median. \\
	Since there are $\lceil n/2 \rceil$ numbers that are smaller than or equal to the median, the probability that any single number we randomly select is smaller than or equal to the median is:
	\[
		\frac{1}{2}
	\]
	Then, we can give a bound on the probability of $E$ and $E^c$:
	\begin{align*}
		Pr(E) &= \left(\frac{\lceil n/2 \rceil}{n}\right) \left(\frac{\lceil n/2 \rceil - 1}{n}\right) \ldots \left(\frac{\lceil n/2 \rceil - k}{n}\right) & \text{sampling with replacement} \\
		& \leq \left( \frac{1}{2} \right)^k & \text{upper bound on the expression above} \\
		\Rightarrow Pr(E) &\leq \left( \frac{1}{2} \right)^k &  \\
		\Rightarrow Pr(E^c) &= 1 - Pr(E) \geq 1 - \left( \frac{1}{2} \right)^k & \text{complement rule} \\
		Pr(E^c) &\geq \frac{2^k - 1}{2^k} & \text{simplifying} \\
	\end{align*}
	Therefore, the probability that the largest number we picked is larger than the median of the $n$ numbers is at least:
	\[
		\frac{2^k - 1}{2^k}
	\]
\end{proof}

\newpage

\section*{Problem 2}
\begin{proof}
	Let $X$ be the RV representing the number of successes we get. Then,
	\begin{align*}
		X &= \sum_{i=1}^{n} X_i & \text{definition of } X \\
		X & \sim Binomial (n, p) & \text{since each coin flip is i.i.d. Bernoulli trial} \\
	\end{align*}
	Examining $\bar{X}$, the average number of successes, we have:
	\begin{align*}
		\bar{X} &= \frac{1}{n} \sum_{i=1}^{n} X_i & \text{definition of } \bar{X} \\
		&= \frac{1}{n} X & \text{definition of } X \\
	\end{align*}
	Using the linearity of expectation, we can compute the expected value of $\bar{X}$:
	\begin{align*}
		E(\bar{X}) &= E(\frac{1}{n} X) & \text{definition of } \bar{X} \\
		&= \frac{1}{n} E(X) & \text{linearity of expectation} \\
		&= \frac{1}{n} (np) & X \sim Binomial(n, p) \\
		&= p & \text{simplifying} \\
	\end{align*}
	Next, we can compute the variance of $\bar{X}$:
	\begin{align*}
		Var(\bar{X}) &= Var(\frac{1}{n} X) & \text{definition of } \bar{X} \\
		&= \left(\frac{1}{n}\right)^2 Var(X) & \text{property of variance} \\
		&= \frac{1}{n^2} (np(1-p)) & X \sim Binomial(n, p) \\
		&= \frac{p(1-p)}{n} & \text{simplifying} \\
	\end{align*}
	Using Chebyshev's inequality, we can give an upper bound on how likely the value of $\bar{X}$ differs from the bias of the coin $p$ by at least $\frac{p}{10}$:
	\begin{align*}
		Pr(|\bar{X} - E(\bar{X})| \geq \frac{p}{10}) &\leq \frac{Var(\bar{X})}{(p/10)^2} & \text{Chebyshev's inequality} \\
		\Rightarrow Pr(|\bar{X} - p| \geq \frac{p}{10}) &\leq \frac{Var(\bar{X})}{(p/10)^2} & \text{substituting in } E(\bar{X}) \\
		&= \frac{\frac{p(1-p)}{n}}{(p/10)^2} & \text{substituting in } Var(\bar{X}) \\
		&= \frac{100(1-p)}{np} & \text{simplifying} \\
	\end{align*}
	Therefore:
	\[
		Pr(|\bar{X} - p| \geq \frac{p}{10}) \leq \frac{100(1-p)}{np}
	\]
\end{proof}

\newpage

\section*{Problem 3}
\begin{claim}
	Let $f$ and $g$ be valid probability distributions defined over the same domain, $S$.
	Then, the convex combination of $f$ and $g$
	\[
		h := \lambda f + (1 - \lambda) g
	\]
	for some $lambda \in [0, 1]$ is also a valid probability distribution.
\end{claim}
\begin{proof}
	To show that $h$ is a valid probability distribution, we need to show that:
	\begin{enumerate}[(i)]
		\item $h(x) \geq 0 \forall x \in S$
		\item $\sum_{x \in S} h(x) = 1$
	\end{enumerate}
	Since the domains of $f$ and $g$ are the same, $h$ is also defined over the same domain, $S$. \\
	For (i), we have:
	\begin{align*}
		h(x) &= \lambda f(x) + (1 - \lambda) g(x) & \text{definition of } h \\
		\sum_{x \in S} h(x) &= \sum_{x \in S} \lambda f(x) + (1 - \lambda) g(x) & \text{applying summation} \\
		&= \sum_{x \in S} \lambda f(x) + \sum_{x \in S} (1 - \lambda) g(x) & \text{linearity of summation} \\
		&= \lambda \sum_{x \in S} f(x) + (1 - \lambda) \sum_{x \in S} g(x) & \text{linearity of summation} \\
		&= \lambda (1) + (1 - \lambda) (1) & \text{since } f \text{ and } g \text{ are valid probability distributions} \\
		&= \lambda + 1 - \lambda & \text{simplifying} \\
		&= 1 & \text{simplifying} \\
	\end{align*}
	Thus, $h$ satisfies (i). \\
	For (ii), we have:
	\begin{align*}
		\lambda f(x) & \geq 0 & \text{since } \lambda \in [0, 1] \text{ and } f(x) \geq 0 \\
		(1 - \lambda) g(x) & \geq 0 & \text{since } (1 - \lambda) \in [0, 1] \text{ and } g(x) \geq 0 \\
		\Rightarrow h(x) &= \lambda f(x) + (1 - \lambda) g(x) \geq 0 + 0 & \text{adding the two inequalities} \\
		&= 0 & \text{simplifying} \\
	\end{align*}
	Thus, $h$ satisfies (ii). \\
	Since $h$ satisfies both (i) and (ii), $h$ is a valid probability distribution.
\end{proof}

\newpage

\section*{Problem 4}
To find the MLE for $p^*$ and $\mu^*$, we will optimize the log-likelihood function:
\[
	Q = \sum_{i=1}^{n} \sum_{j=1}^{k} \gamma_{ij} \left( \log \pi_j + \log f(x_i; \vec{\theta_j}) \right) \\
\]
\[s.t. \sum_{j=1}^{k} \pi_j = 1\]
Using the Langrangian multiplier method, we have:
\begin{align*}
	& \sum_{i=1}^{n} \sum_{j=1}^{k} \gamma_{ij} \left( \log \pi_j + \log f(x_i; \vec{\theta_j}) \right) - \lambda \left( \sum_{j=1}^{k} \pi_j - 1 \right) & \\
	& \pd{\vec{\theta_w}} \left[ \sum_{i=1}^{n} \sum_{j=1}^{k} \gamma_{ij} \left( \log \pi_j + \log f(x_i; \vec{\theta_j}) \right) - \lambda \left( \sum_{j=1}^{k} \pi_j - 1 \right) \right] = 0 & \text{differentiating w.r.t. } \vec{\theta_w} \\
	&= \sum_{i=1}^{n} \gamma_{iw} \pd{\vec{\theta_w}} \log \left( Pr[x_i | \theta_w] \right) & \text{derived in lecture} \\
\end{align*}
For the Geometric distribution, we have:
\[
	Pr[x_i | p] = (1 - p)^{x_i - 1} p
\]
\begin{align*}
	& \sum_{i=1}^{n} \gamma_{iw} \pd{\vec{\theta_w}} \log \left( Pr[x_i | \theta_w] \right) & \text{above} \\ 
	&= \sum_{i=1}^{n} \gamma_{iw} \pd{p} \log \left( (1 - p)^{x_i - 1} p \right) & \text{substituting in } Pr[x_i | p] \\
	&= \sum_{i=1}^{n} \gamma_{iw} \pd{p} \left( (x_i - 1) \log(1 - p) + \log(p) \right) & \text{log property} \\
	&= \sum_{i=1}^{n} \gamma_{iw} \left( - (x_i - 1) \frac{1}{1 - p} + \frac{1}{p} \right) & \text{differentiating} \\
	&= \sum_{i=1}^{n} \gamma_{iw} \left( \frac{(1 - p) - (x_i - 1)p}{p(1 - p)} \right) & \text{combining the two fractions} \\
	&= \sum_{i=1}^{n} \gamma_{iw} \left( \frac{1 - px_i}{p(1 - p)} \right) & \text{simplifying} \\
	&= \frac{1}{p(1 - p)} \sum_{i=1}^{n} \gamma_{iw} (1 - px_i) & \text{factoring out } \frac{1}{p(1 - p)} \\
	& \frac{1}{p(1 - p)} \sum_{i=1}^{n} \gamma_{iw} (1 - px_i) = 0 & \text{setting equal to } 0 \\
	& \Rightarrow \sum_{i=1}^{n} \gamma_{iw} (1 - px_i) = 0 & \text{multiplying both sides by } p(1 - p) \\
	& \Rightarrow \sum_{i=1}^{n} \gamma_{iw} - \sum_{i = 1}^{n} \gamma_{iw} px_i = 0 & \text{distributing the summation} \\
	& \Rightarrow \sum_{i=1}^{n} \gamma_{iw} = p \sum_{i = 1}^{n} \gamma_{iw} x_i & \text{moving the second term to the right side} \\
	& \Rightarrow p^* = \frac{\sum_{i=1}^{n} \gamma_{iw}}{\sum_{i = 1}^{n} \gamma_{iw} x_i} & \text{dividing both sides by } \sum_{i = 1}^{n} \gamma_{iw} x_i \\
\end{align*}

For the Borel distribution, we have:
\[
	Pr[x_i | \mu] = \frac{e^{-\mu x_i} (\mu x_i)^{x_i - 1}}{x_i!}
\]
\begin{align*}
	& \sum_{i=1}^{n} \gamma_{iw} \pd{\vec{\theta_w}} \log \left( Pr[x_i | \theta_w] \right) & \text{above} \\
	&= \sum_{i=1}^{n} \gamma_{iw} \pd{\mu} \log \left( \frac{e^{-\mu x_i} (\mu x_i)^{x_i - 1}}{x_i!} \right) & \text{substituting in } Pr[x_i | \mu] \\
	&= \sum_{i=1}^{n} \gamma_{iw} \pd{\mu} \left( -\mu x_i + (x_i - 1) \log(\mu x_i) - \log(x_i!) \right) & \text{log property} \\
	&= \sum_{i=1}^{n} \gamma_{iw} \left( -x_i + (x_i - 1) \frac{1}{\mu} \right) & \text{differentiating} \\
	& \sum_{i=1}^{n} \gamma_{iw} \left( -x_i + (x_i - 1) \frac{1}{\mu} \right) & \text{setting equal to } 0 \\
	&\Rightarrow \sum_{i=1}^{n} \gamma_{iw} (x_i - 1)\frac{1}{\mu} - \sum_{i=1}^{n} \gamma_{iw} x_i = 0 & \text{splitting the summation and rearranging} \\
	&\Rightarrow \sum_{i=1}^{n} \gamma_{iw} (x_i - 1)\frac{1}{\mu} = \sum_{i=1}^{n} \gamma_{iw} x_i & \text{moving the second term to the right side} \\
	&\Rightarrow \frac{1}{\mu} = \frac{\sum_{i=1}^{n} \gamma_{iw} x_i}{\sum_{i=1}^{n} \gamma_{iw} (x_i - 1)} & \text{dividing both sides by } \sum_{i=1}^{n} \gamma_{iw} (x_i - 1) \\
	&\Rightarrow \mu^* = \frac{\sum_{i=1}^{n} \gamma_{iw} (x_i - 1)}{\sum_{i=1}^{n} \gamma_{iw} x_i} & \text{taking the reciprocal of both sides} \\
\end{align*}
Therefore, the MLEs are:
\[
	p^* = \frac{\sum_{i=1}^{n} \gamma_{iw}}{\sum_{i = 1}^{n} \gamma_{iw} x_i}
\]
\[
	\mu^* = \frac{\sum_{i=1}^{n} \gamma_{iw} (x_i - 1)}{\sum_{i=1}^{n} \gamma_{iw} x_i}
\]

\newpage

\section*{Problem 5}
\begin{proof}
	Let $X$ be the RV representing the total number of successes we get after repeating the experiement $n$ times. Since each coin flip is i.i.d., we have:
	$X \sim Binomial(N, p)$. \\
	Here, $N = nm$, since we have $n$ experiements, each with $m$ coin flips. 
	Let $x_{ij}$ be the outcome of the $j^{th}$ coin flip in the $i^{th}$ experiement, then:
	\[
		x_{ij} = \begin{cases}
			x_{ij} = 1 & \text{if the } j^{th} \text{ coin flip in the } i^{th} \text{ experiement is success} \\
			x_{ij} = 0 & \text{otherwise}
		\end{cases}
	\]
	\begin{align*}
		& x_{ij} & \sim Bernoulli(p) & \text{each coin flip is i.i.d. Bernoulli trial} \\
		&\Rightarrow X_i = \sum_{j=1}^{m} x_ij & \text{definition of } X_i \\
		&\Rightarrow X_i \sim Binomial(m, p) & \text{since each } x_{ij} \text{ is i.i.d. Bernoulli trial} \\
		&\Rightarrow X = \sum_{i=1}^{n} X_i & \text{definition of } X \\
		&\Rightarrow X = \sum_{i=1}^{n} \sum_{j=1}^{m} x_{ij} & \text{substituting in } X_i \\
		&\Rightarrow X  \sim Binomial(nm, p) & \text{since each } X_i \text{ is i.i.d. Binomial trial} \\
	\end{align*}
	Let $k$ be the total number of successes we observed:
	\[
	k = \sum_{i=1}^{n} \sum_{j=1}^{m} x_{ij}
	\]
	Then, we can use the likelihood function to derive the MLE for $p$:
	\begin{align*}
		\mathcal{L}(p) &= Pr[X = k | p] & \text{definition of likelihood function} \\
		&= {nm \choose k} p^k (1 - p)^{nm - k} & X \sim Binomial(nm, p) \\
		\mathcal{L}\mathcal{L}(p) &= \log \left( {nm \choose k} p^k (1 - p)^{nm - k} \right) & \text{taking the log of the likelihood function} \\
		&= \log \left( {nm \choose k} \right) + k \log(p) + (nm - k) \log(1 - p) & \text{log property} \\
		\pd{p} \mathcal{L}\mathcal{L}(p) &= \pd{p} \left[ \log \left( {nm \choose k} \right) + k \log(p) + (nm - k) \log(1 - p) \right] & \text{differentiating} \\
		&= 0 + k \frac{1}{p} - (nm - k) \frac{1}{1 - p} & \text{differentiating} \\
		&= k \frac{1}{p} - (nm - k) \frac{1}{1 - p} & \text{simplifying} \\
		& k \frac{1}{p} - (nm - k) \frac{1}{1 - p} = 0 & \text{setting equal to } 0 \\
		&\Rightarrow \frac{k}{p} = \frac{nm - k}{1 - p} & \text{moving the second term to the right side} \\
		&\Rightarrow \frac{1 - p}{p} = \frac{nm - k}{k} & \text{cross-multiplying} \\
		&\Rightarrow \frac{1}{p} - 1 = \frac{nm - k}{k} & \text{rewriting the left side} \\
		&\Rightarrow \frac{1}{p} = \frac{nm - k}{k} + 1 & \text{adding } 1 \text{ to both sides} \\
		&\Rightarrow \frac{1}{p} = \frac{nm}{k} & \text{combining the right side} \\
		&\Rightarrow p^* = \frac{k}{nm} & \text{taking the reciprocal of both sides} \\
	\end{align*}
	Therefore, the MLE for $p$ is:
	\[
		p^* = \frac{k}{nm}
	\]
	where $k$ is the total number of successes we observed after repeating the experiement $n$ times, each with $m$ coin flips.
\end{proof}

\newpage

\section*{Extra Credit}
\begin{proof}
	Suppose we use our Monte-carlo algorithm from Problem 1, \texttt{pick\_only\_see\_k(pts, k)} to find the median of $n$ numbers by repeatedly sampling $k$ numbers and picking the largest number from the $k$ numbers we sampled. 
	We found that the probability we sample a number larger than the median of the $n$ numbers is at least:
	\[
		\frac{2^k - 1}{2^k}
	\]
	Let $X$ be the RV representing the number of times we have to repeat the sampling process until we pick a number larger than the median. 
	Then,
	\[
		X \sim Geometric(p)
	\]
	where $p$ is the probability of success with each call of \texttt{pick\_only\_see\_k(pts, k)}.
	The amount of times we are expected to call \texttt{pick\_only\_see\_k(pts, k)} is:
	\begin{align*}
		E(X) &= \frac{1}{p} & X \sim Geometric(p) \\
		p &\geq \frac{2^k - 1}{2^k} & \text{from problem 1} \\
		\Rightarrow \frac{1}{p} &\leq \frac{1}{\frac{2^k - 1}{2^k}} & \text{taking the reciprocal of both sides} \\
		\Rightarrow E(X) &\leq \frac{1}{\frac{2^k - 1}{2^k}} & \text{substituting in } E(X) \\
		\Rightarrow E(X) &\leq \frac{2^k}{2^k - 1} & \text{simplifying} \\
	\end{align*}
	Therefore, the expected number of times we have to call \texttt{pick\_only\_see\_k(pts, k)} until we pick a number larger than the median is at most $\frac{2^k}{2^k - 1}$.
\end{proof}
\end{document}